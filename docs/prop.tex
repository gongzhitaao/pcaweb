\RequirePackage[l2tabu, orthodox]{nag}

\documentclass[letterpaper]{scrartcl}

\usepackage[T1]{fontenc}
%\usepackage[sc]{mathpazo}
%\usepackage{charter}
\usepackage{helvet}
\usepackage{microtype}

%\usepackage{algorithm, algpseudocode}
%\usepackage{float}
%\usepackage{amssymb, amsmath, amsthm}
%\usepackage{mathtools}

%% for displaying codes, minted is perferred.
%% see table.tex for examples
%\usepackage{listings}
%\usepackage{minted}

\usepackage{graphicx}

%% remove space between list
\usepackage{enumitem}
%\setlist{nolistsep}
\setlist{noitemsep}

%\usepackage{biblatex}
%\bibliography{default.bib}

\usepackage{url}
\usepackage[colorlinks,linkcolor=red,anchorcolor=blue,%
  citecolor=green,unicode={true}]{hyperref}

%\setlength{\textwidth}{6.5in}
%\setlength{\textheight}{9.0in}
%\setlength{\topmargin}{-.5in}
%\setlength{\oddsidemargin}{-.0600in}
%\setlength{\evensidemargin}{.0625in}

%\newcommand{\secref}[1]{Section~\ref{#1}}

%\newcommand{\doublespace}{\baselineskip0.34truein}
%\newcommand{\singlespace}{\baselineskip0.16truein}
%\newcommand{\midspace}{\baselineskip0.24truein}
%\newcommand{\midplusspace}{\baselineskip0.26truein}

\title{Personal Computer Applications}

\subtitle{COMP6000 Semester Project Proposal}

\author{Zhitao Gong, Xing Wu}

\begin{document}

\maketitle

\section{Introduction}
{\em Personal Computer Applications at Auburn} is a website aiming at
inspiring more students to choose careers in IT and computing.

It contains information about related courses, e.g. schedules,
syllabus, instructors, etc. and also various updates about summer
camps and etc.

\section{Proposed Work}

The original site was implemented using ASP.NET.  But considering the
website's aims, structures and features, it's more of a static
website.  So we propose to re-implement it using
\href{https://github.com/mojombo/jekyll}{Jekyll}, a static site
generator, with the following goals:

\begin{itemize}
\item Restructure the website to make it more user friendly.
\item Platform independent.
\item Easy to maintain and update.
\end{itemize}


\end{document}

%%  LocalWords:  tabu
